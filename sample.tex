\documentclass[12pt,a4paper,titlepage]{article}
%\documentclass[12pt, a4paper, UTF8]{ctexart}
\usepackage{CJK}

\usepackage{ctex}
\usepackage{lipsum}     %随机生成文本的宏包
\usepackage{geometry}   %设置页边距的宏包
\usepackage{titlesec}   %设置页眉页脚的宏包
\usepackage{setspace}   %行距
\usepackage{fancyhdr}	   %页眉页脚  
%\usepackage{titlesec}   %设置子标题字体
\usepackage{titlesec}

\titleformat{\section}%设置section的样式
{\center\zihao{-2}\bfseries}%右对齐,小二号字,加粗
{\thesection .\quad}%标号后面有个点
{0pt}%sep label和title之间的水平距离
{}%标题前没有内容

\titleformat{\subsection}%设置section的样式
{\raggedright\zihao{-3}\bfseries}%右对齐,小三号字,加粗
{\thesection .\quad}%标号后面有个点
{0pt}%sep label和title之间的水平距离
{}%标题前没有内容

\titleformat{\subsection}%设置section的样式
{\raggedright\zihao{4}\bfseries}%右对齐,四号字,加粗
{\thesection .\quad}%标号后面有个点
{0pt}%sep label和title之间的水平距离
{}%标题前没有内容 


\setCJKmainfont{宋体}
\pagestyle{fancy}
\fancyhf{}
\setlength{\baselineskip}{18pt}
\geometry{left=3cm,right=2.5cm,top=2.5cm,bottom=2.5cm}  %设置 上、左、下、右 页边距

\cfoot{\zihao{5} \thepage}
\fancyhead[CO,CE]{\zihao{4} 安徽工业大学毕业设计(论文)}


\title{基于深度学习的行人检测算法探究与复现}
\author{吴若晨}
\date{\today}


\begin{document}
\maketitle

\renewcommand{\abstractname}{\zihao{3} 摘要}
\begin{abstract}
相较于传统计算机视觉算法,通过手工合理提取特征,再进行分类的做法,深度学习由于其强大的自动提取大量特征的特性和良好的泛化性和鲁棒性,在目标识别分类方向取得了巨大突破,并由此掀起了深度学习的狂潮。而深度学习技术,在目标检测方向的研究同样在近几年取得了激动人心的成果。本科设旨在对今年来有重大突破意义的目标检测算法进行探究和复现,并利用迁移学习的原理,将其迁移至行人检测这个特定的应用场景中来。
\begin{center}
{\textbf{关键词}:卷积神经网络,物体检测,迁移学习,行人检测}
\end{center}
\end{abstract}

\renewcommand{\abstractname}{\zihao{3} Abstract}
\begin{abstract}
Compared with the traditional computer vision algorithms, through reasonable manual feature extraction, classification and practice of deep learning due to its characteristics of strong automatic extraction of large feature and good generalization ability and robustness, has made a great breakthrough in the direction of target recognition and classification, and from this set off a frenzy of deep learning. While deep learning technology, the research on the direction of target detection has also achieved exciting results in recent years. The purpose of this research is to explore and reproduce the major breakthrough detection algorithms this year, and to transfer to the specific application scenario of pedestrian detection based on the principle of transfer learning.
\begin{center}
{\textbf{keywords}: CNN, object detection, tranform learning, pedestrain detection}
\end{center}
\end{abstract}

\renewcommand{\contentsname}{目录}
\tableofcontents
\newpage

\section{绪论}
\subsection{研究背景及意义}
\subsubsection{研究背景}
行人检测作为计算机视觉中的一个经典的主题,其在学术界和工业界一直有着广泛的研究和应用。在安防领域,行人检测可以应用在安防监控系统,可以精准定位行人的位置所在,分析行人行为,可以辨别一个人行为是否异常,达到防范的目的,同时,通过联合多个安防摄像头传回的数据,在行人检测算法的基础上加以改进,可以实现定位和追踪一个人的路径,这为警方追踪嫌疑犯提供了技术上的便利。在智能交通领域,通过在路口的监控头,可以智能检测和统计,各时段行人流量的变化,为后续各种优化交通决策提供基础。在自动驾驶领域,通过车载摄像头,行人检测算法可以判断四周环境,定位行人位置,为自动驾驶系统决策提供关键的辅助信息。总之,随着行人检测在安防,智能交通,自动驾驶等领域大量需求的出现,快速且精准的行人检测算法研究热度一直高居不下。考虑到行人外观易受穿着,尺度,遮挡,姿态,视角等影响,同时行人检测也是行人追踪,行人重识别等问题的基础,所以行人检测仍然是计算机视觉领域一个颇具挑战性和价值性的研究话题。
\subsubsection{研究意义}
相较于传统计算机视觉算法,通过手工合理提取特征,再进行分类的做法,深度学习由于其强大的自动提取大量特征的特性和良好的泛化性和鲁棒性,在目标识别分类方向取得了巨大突破,并由此掀起了深度学习的狂潮。而深度学习技术,在目标检测方向的研究同样在近几年取得了激动人心的成果。本科设旨在对今年来有重大突破意义的目标检测算法进行探究和复现,并利用迁移学习的原理,将其迁移至行人检测这个特定的应用场景中来

\subsection{国内外论文综述}
\subsubsection{国内相关行人检测研究}
几篇论文
\subsubsection{国外相关行人检测研究}
几篇论文

\subsection{研究方法与内容}
\subsubsection{研究方法}
深度卷积网络,迁移学习
\subsubsection{研究内容}
具体算法

\section{算法}
\subsection{传统算法}
Hog+SVM
\subsection{faster-RCNN}
\subsubsection{算法原理}
region proposal + classifier
\subsubsection{实验表现}
PASCAL VOC 71.2 mAP   15FPS
\subsection{YOLO}
\subsubsection{算法原理}
a unified convolutional nerual network.
\subsubsection{实验表现}
PACAL VOC 63.0 mAP    45FPS
\subsection{SSD}
\subsubsection{算法原理}
\subsubsection{实验表现}


\section{算法的实际场景应用}

\section{附录}

\end{document}

